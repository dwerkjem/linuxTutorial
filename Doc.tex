\documentclass{book}

\usepackage[utf8]{inputenc}
\usepackage{amsmath}
\usepackage{graphicx}
\usepackage{hyperref}
\usepackage{bookmark}
\usepackage{listings}
\usepackage{xcolor} % Use xcolor instead of color for more options

% Define custom colors
\definecolor{codegreen}{rgb}{0,0.6,0}
\definecolor{codegray}{rgb}{0.5,0.5,0.5}
\definecolor{codepurple}{rgb}{0.58,0,0.82}
\definecolor{codemauve}{rgb}{0.58,0,0.82}
\definecolor{codeorange}{rgb}{0.8,0.33,0}
\definecolor{backcolour}{rgb}{0.95,0.95,0.92}

% Define Bash language for listings
\lstdefinelanguage{bash}{
    keywords={sudo, mv, cp, rm, ls, cd, mkdir, rmdir, chown, chmod, touch, echo, exit, apt, apt-get, dpkg, systemctl, service, useradd, usermod, passwd, nano, vim, git, clone, push, pull, commit},
    keywordstyle=\color{blue}\bfseries,
    ndkeywords={if, then, else, fi, for, in, do, done, exit, return, set, function},
    ndkeywordstyle=\color{codepurple}\bfseries,
    identifierstyle=\color{black},
    sensitive=true,
    comment=[l]{\#},
    commentstyle=\color{codegreen}\ttfamily,
    stringstyle=\color{codeorange}\ttfamily,
    morestring=[b]',
    morestring=[b]"
}

% Define the style for Bash listings
\lstdefinestyle{bashstyle}{
    backgroundcolor=\color{backcolour},
    basicstyle=\ttfamily\footnotesize,
    keywordstyle=\color{blue}\bfseries,
    ndkeywordstyle=\color{codepurple}\bfseries,
    identifierstyle=\color{black},
    commentstyle=\color{codegreen}\ttfamily,
    stringstyle=\color{codeorange}\ttfamily,
    stepnumber=1,
    numbersep=10pt,
    tabsize=2,
    breaklines=true,
    breakatwhitespace=false,
    showspaces=false,
    showstringspaces=false,
    showtabs=false,
    captionpos=b,
    language=bash
}

\lstset{style=bashstyle}

\begin{document}

\chapter{How to use this tutorial}
\section{Introduction}
This tutorial was made to aid my friend in installing and configuring Linux on his laptop. It is made to cover edge cases and to be as simple as possible. This tutorial is made for beginners and is not meant to be a comprehensive guide to Linux. I hope someone finds this tutorial useful and that it helps them in their journey to learn Linux.

\section{How to use this tutorial}
The tutorial is divided into chapters. Each chapter covers a specific topic. The chapters are meant to be read in order. The tutorial is meant to be read on a phone or a laptop while you are installing Linux on your computer.\par

A chapter will start with a brief introduction to the topic and then will have a list of steps that you need to follow. The steps are meant to be followed in order. Each step will have a brief explanation of what you need to do. The steps are meant to be simple and easy to follow. If you have any questions or need help, feel free to ask me. I will try to help you as much as I can.\par

\chapter{What and why of Debian and Linux}
\section{What and why Debian?}
Debian is a free operating system (OS) for your computer. An operating system is a set of programs that manage the hardware and software of your computer. Debian is fundamentally different from Windows and macOS. Debian is made by a community of people who work together to create a free and open-source operating system while Windows and macOS are made by companies. The companies naturally want to make a profit, so they do not share their code with the public and are often in my opinion anti consumer and anti-privacy. Debian is made by people who want to make a good operating system that is free as in free to do as you want to it and free as in freedom. Debian is made to be secure, stable, and easy to use unlike Windows and macOS which are close source and are hard to audit for security vulnerabilities while Debian is open source and can be audited by anyone. Debian is made to be easy to use and is made to be used by anyone. There are three main distributions of Linux that most people use Debian, Fedora, and Arch most other distributions are spinoffs of these three. Debian is the most stable and secure of the three and is the one that I recommend for beginners. Here is a list of reasons why you should use Debian: 
\begin{itemize}
    \item Debian is free and open-source
    \subitem unlike Windows and macOS which are closed source
    \item Debian is secure
    \subitem Debian is made to be secure and is made to be used by anyone
    \item Debian is stable
    \subitem Debian is made to be stable and not bleeding edge like Arch Linux
    \item Debian is easy to use 
    \subitem Debian is made to be easy to use and is made to be used by anyone unlike Arch Linux which is made for advanced users 
    \item Debian is well documented and supported
    \subitem Debian has a large community of users and developers who can help you if you have any questions or need help unlike Fedora which is made by a company and is not as well documented or supported 
    \subitem All ubuntu packages are based on Debian packages so you can use Ubuntu packages on Debian
\end{itemize}

For all these reasons I recommend Debian for beginners. Debian is made to be easy to use and is made to be used by anyone. Debian is made to be secure, stable, and easy to use. Debian is free and open-source and is made to be used by anyone. Debian is well documented and supported and has a large community of users and developers who can help you if you have any questions or need help. Debian is the best distribution of Linux for beginners and is the one that I recommend for beginners.

\section{Vocabulary}
\begin{itemize}
    \item Operating System (OS) - A set of programs that manage the hardware and software of your computer
    \item Bleeding edge - The latest and greatest software that is not stable
    \item Spinoff - A distribution of Linux that is based on another distribution of Linux
    \item distribution - A version of Linux that is made by a community of people who work together to create a free and open-source operating system
    \item open-source - Software that is free to use and can be audited by anyone
\end{itemize}

\section{Prerequisites}
Before you start installing Debian, you need to have a few things ready. Here is a list of things that you need to have before you start installing Debian:
\begin{itemize}
    \item A computer with at least 2GB of RAM and 20GB of free space and a 64-bit processor
    \item A USB drive with at least 4GB of space
    \item A keyboard and mouse
    \item An internet connection (optional but recommended)
    \item A phone or a laptop to read this tutorial
    \item A phone or a laptop to download the Debian ISO file and to create a bootable USB drive
\end{itemize}

\section{Downloading Debian} 
The first step in installing Debian is to download the Debian ISO file. The Debian ISO file is a file that contains the Debian operating system. You can download the Debian ISO file from the Debian website. Here is how you can download the Debian ISO file: https://www.debian.org/

\section{Creating a bootable USB drive}

The next step in installing Debian is to create a bootable USB drive. A bootable USB drive is a USB drive that contains the Debian ISO file and can be used to install Debian on your computer. You can create a bootable USB drive using a program called Rufus. Rufus is a free and open-source program that can be used to create bootable USB drives. You can download Rufus from the Rufus website. Here is how you can create a bootable USB drive using Rufus:

\begin{itemize}
    \item Download Rufus from the Rufus website
    \item Insert the USB drive into your computer
    \item Open Rufus
    \item Select the USB drive that you want to use
    \item Select the Debian ISO file that you downloaded
    \item Click on the start button
    \item select DD mode and click on start
    \item Wait for Rufus to finish creating the bootable USB drive
    \item Once Rufus has finished creating the bootable USB drive, you can remove the USB drive from your computer
\end{itemize}

\section{Booting from the USB drive}

The next step in installing Debian is to boot from the USB drive. To boot from the USB drive, you need to restart your computer and enter the BIOS or UEFI settings. The BIOS or UEFI settings are the settings that control how your computer boots. You can enter the BIOS or UEFI settings by pressing a key on your keyboard when your computer starts up. The key that you need to press depends on your computer. Here is how you can enter the BIOS or UEFI settings:

\begin{itemize}
    \item Restart your computer
    \item Press the key that you need to press to enter the BIOS or UEFI settings (usually F2, F10, or Del)
    \item Select the USB drive as the boot device
    \item Save the settings and exit the BIOS or UEFI settings
    \item Your computer will now boot from the USB drive
    \item You will see the Debian installer
\end{itemize}

\section{Installing Debian (installer)}

The next step in installing Debian is to install Debian using the Debian installer. The Debian installer is a program that guides you through the installation process. You can use the Debian installer to install Debian on your computer. Here is how you can install Debian using the Debian installer:


\begin{itemize}
    \item Select the language that you want to use
    \item Select your keyboard layout 
    \subitem If you are not sure, select the default option
    \item select your network connection 
    \subitem most likely you will want to use a wired connection if you have one
    \subitem if you do not have a wired connection, you can select the option to use a wireless connection
    \item Enter your hostname
    \subitem The hostname is the name of your computer on the network
    \subitem You can enter any name that you want
    \item Enter your domain name
    \subitem The domain name is the name of your network 
    \subitem You can skip this step if you do not have a domain name
    \item Enter your root password
    \subitem The root password is the password that you will use to log in as the root user AKA the superuser Administrator on Windows
    \subitem You can enter any password that you want but make sure that it is secure and that you remember it 
    \subitem \textbf{THERE IS NO WAY TO RECOVER A LOST ROOT PASSWORD. IF YOU FORGET IT, YOU WILL HAVE TO REINSTALL DEBIAN.}
    \item Enter your desired name this is not your username but the name that you want to use for your account 
    \item Enter your username
    \item Enter your password
    \subitem The password is the password that you will use to log in to your account
    \subitem You can enter any password that you want but make sure that it is secure and that you remember it
    \item Select your time zone
    \subitem The time zone is the time zone that you are in do not worry about this if you're always moving enter the time zone that you are in most of the time 
    \item Partition your disk
    \subitem if you want to use the entire disk for Debian, you can select the option to use the entire disk
    \subitem if you want to partition the disk yourself, you can select the option to partition the disk yourself
    \subitem if you are not sure, you can select the default option
    \subitem select all files in one partition
    \item Wait for Debian to finish partitioning the disk and installing the base system
    \item configure the package manager
    \subitem select the mirror country that is closest to you 
    \subitem select the mirror that is closest to you or select the default option
    \item select no proxy
    \item select the software that you want to install
    \subitem if you are not sure, select the following options:
    \subsubitem Debian desktop environment
    \subsubitem web server
    \subsubitem ssh server
    \subsubitem standard system utilities
    \subsubitem \textbf{Do not press continue yet}
    \item selecting your desktop environment
    \subitem this is the desktop environment that you will use to interact with Debian it determines how Debian looks and feels if you are not sure, select KDE and GNOME both, and you will be able to choose between them when you log in
    \subitem Gnome looks not as good as KDE but is more stable and is the default desktop environment for Debian 
    \subitem KDE looks Far better than Gnome but is less stable and is not the default desktop environment for Debian I recommend using KDE if you have a good computer and Gnome if you have a bad computer for my friend I recommend using KDE even on a laptop 
    \subitem press continue
    \item wait for Debian to finish installing the software
\end{itemize}
\chapter{Configuring Debian}

You will need to do some system configuration after installing Debian to achieve optimal performance style and functionality. This chapter will guide you through the process of configuring Debian to your liking.

\section{Adding terminal shortcuts}

Press \texttt{Ctrl+Alt+T} to open a terminal. If this shortcut does not work, you can add a custom shortcut to open a terminal. Search for \texttt{keyboard} in the application menu and open the keyboard settings. Click on the \texttt{Shortcuts} tab and then click on the \texttt{Custom Shortcuts} option. Click on the \texttt{+} button to add a new shortcut. In the command field, type \texttt{gnome-terminal} and in the shortcut field, press the keys you want to use to open the terminal. I use \texttt{Ctrl+Alt+T} to open the terminal.

\section{Adding your user to the sudo group}

To add your user to the sudo group, open a terminal and type the following command:

\begin{lstlisting}
su

usermod -aG sudo {username}
\end{lstlisting}

Replace \texttt{\{username\}} with your username.

\section{Proprietary drivers}

Debian does not include proprietary drivers by default. If you have hardware that requires proprietary drivers, you will need to install them manually. To install proprietary drivers, open the \texttt{Konsole} terminal and type the following command:

\begin{lstlisting}
sudo tee /etc/apt/sources.list > /dev/null <<EOF
deb http://deb.debian.org/debian bookworm main contrib non-free non-free-firmware
deb-src http://deb.debian.org/debian bookworm main contrib non-free non-free-firmware

deb http://deb.debian.org/debian-security bookworm-security main contrib non-free non-free-firmware
deb-src http://deb.debian.org/debian-security bookworm-security main contrib non-free non-free-firmware

deb http://deb.debian.org/debian bookworm-updates main contrib non-free non-free-firmware
deb-src http://deb.debian.org/debian bookworm-updates main contrib non-free non-free-firmware
EOF

sudo apt update
sudo apt install nvidia-driver # use the appropriate driver for your hardware
\end{lstlisting}

Replace \texttt{nvidia-driver} with the appropriate driver for your hardware.

\section{installing Discord}

To install Discord, open the Discover application and add Snap support. To add Snap support, go to the Snap Store and open settings. Click on the \texttt{Enable Snapd} option. After enabling Snapd, restart your computer. Open the Discover application and search for Discord. Click on the installation button to install Discord. Steam and Spotify should be installed in the same way. 

\section{Installing extra software}

To install most software, you can use the \texttt{apt} package manager. To install software, open the \texttt{Konsole} terminal and type the following command:

\begin{lstlisting}
sudo apt update
sudo apt install {package-name}
\end{lstlisting}

Replace \texttt{\{package-name\}} with the name of the package you want to install.

recommended software to install:

\begin{lstlisting}
sudo apt install neofetch htop gparted vlc vim git curl wget snapd netselect-apt gnome-tweaks openvpn qbittorrent zsh zsh-syntax-highlighting zsh-autosuggestions 
\end{lstlisting}

\section{Replace the default shell}

The default shell in Debian is \texttt{bash}. You can replace the default shell with \texttt{zsh}. To install \texttt{zsh}, open the \texttt{Konsole} terminal and type the following command:

\begin{lstlisting}
    sh -c "$(curl -fsSL https://raw.githubusercontent.com/ohmyzsh/ohmyzsh/master/tools/install.sh)"
\end{lstlisting}

After installing \texttt{zsh}, you will need to change the default shell. To change the default shell, open the \texttt{Konsole} terminal and type the following command:

\begin{lstlisting}
chsh -s $(which zsh)
\end{lstlisting}

\section{Change the default terminal}

The default terminal in Debian is \texttt{Konsole}. You can replace the default terminal with \texttt{wezterm}. To install \texttt{wezterm}, open the \texttt{Konsole} terminal and type the following command:

\begin{lstlisting}
curl -fsSL https://apt.fury.io/wez/gpg.key | sudo gpg --yes --dearmor -o /etc/apt/keyrings/wezterm-fury.gpg
echo 'deb [signed-by=/etc/apt/keyrings/wezterm-fury.gpg] https://apt.fury.io/wez/ * *' | sudo tee /etc/apt/sources.list.d/wezterm.list
sudo apt update
sudo apt install wezterm
\end{lstlisting}

After installing \texttt{wezterm}, you will need to change the default terminal. To change the default terminal, rebind the terminal shortcut to run \texttt{darktile} instead of \texttt{Konsole}.

\section{Configuring the terminal}

To configure the terminal, open the terminal and type the following command:

\begin{lstlisting}
nano ~/.wezterm.lua
\end{lstlisting}


Add the following configuration to the \texttt{.wezterm.lua} file:

\begin{lstlisting}

    local wezterm = require 'wezterm'

    -- This will hold the configuration.
    local config = wezterm.config_builder()
    
    -- This is where you actually apply your config choices
    
    -- For example, changing the color scheme:
    config.color_scheme = 'LiquidCarbonTransparent'
    config.hide_tab_bar_if_only_one_tab = true
    
    -- Set background transparency
    config.window_background_opacity = 0.8  -- Adjust this value for desired transparency (0.0 to 1.0)
    -- config.window_background_image = '/path/to/picture'
    -- and finally, return the configuration to wezterm
    return config
    
\end{lstlisting}


\end{document}
